\subsubsection{Bootstrapping Error
Correction}\label{sec:intro_related_work}
Techniques to correct bootstrapping errors or, more generally, to
minimize distributional shift in offline reinforcement learning
involve either \textit{policy constraint} or
\textit{uncertainty-based} methods \citep{levine2020offline}.

The $Q$ functions is evaluated on the same states that it is
trained on. Therefore, only the action inputs across states can be out
of distribution in the training process. \textit{Policy constraint}
methods address this issue by bounding the distribution over actions
used for the computation of the TD-targets,
$\pi\left(a^\prime\middle|s^\prime\right)$, to stay in the proximity
of the one induced by the behavior policy,
$\pi_\beta\left(a^\prime\middle|s^\prime\right)$. In this way, the $Q$
function regression is driven by target values for which enough
reliable information is found in $\pi_\beta$.
The difference
between these techniques resides in the metrics they employ to define
distributional proximity.
For example, Batch-Constrained deep Q-Learning (BCQ)
\citep{pmlr-v97-fujimoto19a} trains a generative model -- a variational
auto-encoder (VAE) \citep{kingma2013auto} -- to produce actions which
are likely given the data in $\mathcal{B}$, then it perturbs them to
increase diversity in a constrained manner, and finally selects the
maximum $Q$-value over these artificial actions. By substituting the
maximum over all possible actions at the next state, involved in the
computation of the TD-target, with the maximum over actions likely
under $\pi_\beta$ BCQ ensures that the learned policy $\pi$ is
centered around $\pi_\beta$, hence that the $Q$ function is not
queried on OOD actions with resulting bootstrapping
error. Bootstrapping Error Accumulation Reduction Q-Learning (BEAR-QL)
\citep{kumar2019stabilizing} also follows the intuition of placing
constraints on the learned action distribution, however it achieves so
with fewer restrictions. This makes it more viable for $\pi$ to
improve on $\pi_\beta$, a capability hindered by the aggressive
constraint of BCQ.\ Instead of requiring the learned policy to be
close in distribution to $\pi_\beta$, BEAR demands a \textit{support
constraint} \citep{kumar}. This loose condition means that the
learned policy must place non-zero probability on all those actions
that have non-negligible probability according to the behavior policy.
This precaution allows BEAR to improve over suboptimal, even random
off-policy data, whereas BCQ would learn a policy close to uniform
\citep{kumar2019stabilizing}.
