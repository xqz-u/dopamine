% FIXME there is a break-line somewhere after the second paragraph that
% appears idk why
% TODO read all again once the other subsections are defined, and
% evaluate what is written here. Should it be split up? Is the story
% coherent, are there no logical gaps and - especially, this is a
% mistake you tend to make - are you giving too much for granted without
% explaining it?
\section{Introduction}\label{sec:introduction}
% TS: advancements in ML are data-driven, we need the same for RL - can
% achieve it with offline RL
% NOTE could add some more references?
% NOTE shorten or split into sub-paragraphs, ca. 100w each
In the past decade, machine learning methods have encountered
major success over a wide number of real-world applications, ranging
from Computer Vision to Natural Language Processing tasks.
Much of the progress in these areas can be attributed to the
development of \textit{data-driven}, scalable learning methods. In
fact, although advancements in models and architectures are an integral
part of this success story, the learning techniques they employ are
well-founded and understood, and major improvements in model
performance stem from the availability of large and diverse training
datasets.
However, such data-driven methods do not map to the Reinforcement
Learning (RL) framework equally well. RL involves sequential decision
making problems where the best behavior strategy is learned through
active interaction with the environment. This naturally
\textit{online} learning paradigm prevents effective exploitation of
the rich \textit{offline} datasets, which is at the base of the
success of supervised learning methods.
Moreover, data collection for complex real-world RL applications such as
autonomous driving or healthcare support systems can often be
expensive or hazardous. Developing safe and capable
offline RL agents has therefore a great appeal: by efficiently
learning from large amounts of data, we could create ``generalizable
and powerful \textit{decision making engines}''
\citep{levine2020offline} that can aid in solving many real-world open
problems \citep{levine2020offline}.

% TS: offline RL entails reasoning about what has not happened
% (counterfactual queries); this is problematic for standard ML and
% creates distribution shift DS
In offline RL, an agent learns a policy from a static dataset of
logged experiences produced by a \textit{behavior policy}.
This differs from classical online RL, where an agent can actively
collect new experience by interacting with its environment
\citep{sutton2018reinforcement}. For an offline agent to
improve, it is therefore crucial that it is able to partake in
counterfactual
reasoning to accurately estimate what would happen if it took a
decision different from the one in the training dataset
\citep{levine2020offline}. By contrast, an online RL agent could
explore and learn on its own the effects of a decision
different than the one previously chosen in the same situation. This
additional constraint of offline RL is
problematic, and it exceeds the capabilities of current machine learning
methods that use expressive function approximators (neural networks)
to generalize across examples. For one, it violates the assumption of
independent and
identically distributed data (i.i.d.) that standard supervised
learning algorithms rely on: an offline RL agent may be trained under
one distribution but tested under a different one.
In addition, an offline agent must be able to reason differently from
the data-generating policy in order to produce novel -- and possibly
favorable -- courses of actions; this requirement breaks the i.i.d.\
assumption too.
The mismatch between the behavior policy-induced distribution and the
one learned during training is called \textit{distributional shift},
and it presents a fundamental challenge for the efficacy of offline RL
\citep{levine2020offline}.

% TS: DS affects online and offline off-policy algos, especially in
% the form of bootstrapping error
Distributional shift generally affects off-policy RL algorithms. These
are methods that learn about a \textit{target policy} using a
different \textit{behavior policy} \citep{sutton2018reinforcement},
such as the popular Q-learning agent \citep{watkins1992q}. Due to
their ability to learn from data generated by another policy,
off-policy algorithms naturally lend themselves to offline RL.\@ It is
well known that off-policy methods exhibit high variance
\citep{sutton2018reinforcement}; moreover,
off-policy algorithms which employ a maximization operation in the
bootstrapping step, such as Q-learning,
are prone to overoptimistic value estimates
\citep{thrun1993issues}. In the offline setting, this form of
\textit{bootstrapping error} results in the selection of actions that
lie outside of the training data distribution
\citep{kumar2019stabilizing}, and since no ground truth information to
estimate their value is available - the dataset is fixed - they
disrupt the training process and drive it towards regions of
uncertainty; a more in-depth account of the bootstrapping error
follows in Section~\ref{sec:BE}.

% NOTE look at commented part at end of paragraph for another
% formulation of RQ
% TS: ensemble methods address the bootstrapping error, off-policy and
% offline generally; do the results related to standard Deep
% Q-Learning with ensembles transfer to other value-based methods,
% such as DQV family?
To address distributional shift and bootstrapping errors in
off-policy learning, many techniques found in the literature employ
ensemble-based methods (\citeauthor{osband2016deep},
\citeyear{osband2016deep}; \citeauthor{anschel2017averaged},
\citeyear{anschel2017averaged}; \citeauthor{pmlr-v97-fujimoto19a},
\citeyear{pmlr-v97-fujimoto19a}, appendix D.2).
% \citep{osband2016deep, anschel2017averaged,
%   pmlr-v97-fujimoto19a}.
Most relevant for this research,
ensembling methods have demonstrated a successful approach in
offline RL, as seen in the REM agent by
\citet{agarwal2020optimistic}. This
paper investigates to which extent results concerning bootstrapping
error prevention based on ensembled versions of Q-learning transfer to
other model-free, value-based RL algorithms in the context of offline
RL.\@ In particular, we will inquire whether benefits concerning
bootstrapping error reduction stemming from simple ensembling methods
apply to the deep RL agents of the DQV algorithmic family
\citep{sabatelli2020deep}.
% also apply to RL algorithms that already take steps to minimize it,
% such as the DQV algorithmic family \citep{sabatelli2020deep}.

\subsection{Background}
The following subsections will introduce some definitions and prior
knowledge needed to understand this paper.

% NOTE expressed the set of actions w/out a time subscript on the
% actions assuming that all actions are available at all states
% NOTE using R(s,a) as reward signal instead of R_t in order to avoid
% explaining it is a random variable, the explanation seems more
% straightforward in this way to me
% TODO should I write more of the presented equations in a an \equation
% environment? Check that later, keep it tightly spaced for now
\subsubsection{Reinforcement Learning}\label{sec:RL_BG}
% TS: RL general elements
Reinforcement Learning seeks to solve a Markov Decision Process
(MDP) ($\mathcal{S},\mathcal{A},p,R$). In
RL, an agent interacts with an environment at
discrete time steps $t=0,1,2,3,\ldots$. At each time step $t$, the
agent receives a representation of the environment $s_t \in
\mathcal{S}$, on which it can perform some action $a_t \in
\mathcal{A}(s)$ that determines its transition to a new state
$s_{t+1}$ according to the dynamics model
$p\colon\mathcal{S}\times\mathcal{S}\times\mathcal{A}\rightarrow
\left[0,1\right],p\left(s^\prime\mid
  s,a\right)\doteq\Pr\left\{s^\prime=s_{t+1}\mid
  s=s_t,a=a_t\right\}$. Following this transition, the agent receives
reward $r_{t+1}$ from a reward function
$R\colon\mathcal{S}\times\mathcal{A}\rightarrow\mathbb{R},
r_{t+1}=R\left(s_t,a_t\right)$. The goal of a RL agent is then to find
a mapping from states to action probabilities, called a \textit{policy}
$\pi\colon\mathcal{S}\times\mathcal{A}\rightarrow\left[0,1\right],
\pi\left(a\mid s\right)=\Pr\left\{a_t=a,s_t=s\right\}$, which
maximizes the \textit{expected return} $G_t\doteq
\sum_{t=0}^{\infty}\gamma^{t}R\left(s_t,a_t\right)$ where
$\gamma\in\left[0,1\right]$ is a discount factor used to scale the
importance of future rewards. Each policy $\pi$ has a matching
\textit{state value function}
$V^{\pi}\left(s\right)=\mathbb{E}\left[G_t\mid s=s_t\right]$, which
indicates the expected return obtained starting from state $s$ therefore
following $\pi$. This function can also be expressed in terms of
state-action pairs as a \textit{state-action value function}
$Q^{\pi}\left(s,a\right)=\mathbb{E}\left[G_t|s=s_t,a=a_t\right]$,
indicating the expected return taking action $a$ in state $s$ and
consequently following $\pi$. Altogether, the RL optimization problem aims
to achieve a policy $\pi^*$ characterized by the \textit{optimal Q
  value function}
$Q^*\left(s,a\right)\doteq\max_{\pi}Q^{\pi}\left(s,a\right)$ for all
$s\in\mathcal{S}$ and $a\in\mathcal{A}\left(s\right)$, whose solution
is provided by the Bellman optimality equation
\begin{equation}
\begin{aligned}
Q^*\left(s_t,a_t\right)=\mathbb{E}\bigg[&R\left(s_t,a_t\right)+\\ &\gamma
  \max_{a\in\mathcal{A}\left(s\right)}Q^*\left(s_{t+1},a\right)\bigg\vert
  s_t=s,a_t=a \bigg]
\end{aligned}
\end{equation}
\citep{bellman1957dynamic}. Note that the latter can
also be expressed as the \textit{optimal state value function}
$V^*\left(s\right)$ replacing the optimal $Q$ value estimate at
the next state
$\max_{a\in\mathcal{A}\left(s\right)}Q^*\left(s_{t+1},a\right)$ by
$V^*\left(s_{t+1}\right)$.

% TS: Deep RL and Q-learning
$Q^*\left(s,a\right)$ and $V^*\left(s\right)$ can both be learned by
Temporal Difference (TD) learning
\citep{sutton1988learning}. Q-learning is probably the most popular TD
method; it learns the state-action value function using the update
rule
\begin{equation}
Q\left(s_t,a_t\right)\leftarrow
Q\left(s_t,a_t\right)+\alpha\left[R\left(s_t,a_t\right)+y^{\scriptscriptstyle
TD}_{\scriptscriptstyle QL}\right],
\end{equation}
where
\[
y^{\scriptscriptstyle TD}_{\scriptscriptstyle
QL}=\gamma\max_{a\in\mathcal{A}\left(s\right)}Q\left(s_{t+1},a\right)-Q\left(s_t,a_t\right)
\]
\citep{watkins1992q}. When the state space $\mathcal{S}$ is large and
high dimensional, the $Q$ function is approximated by expressive
neural networks and we talk of Deep Reinforcement Learning
(DRL). DRL agents such as Deep Q-Learning (DQN)
\citep{mnih2013playing} have attained super-human performance on a
range of complex tasks such as the ALE benchmark
\citep{bellemare2013arcade}. DRL algorithms generally adapt the $Q$
function to include a neural network parameterized by $\theta$, and
reformulate the standard DQN update rule to be a differentiable loss
function
\begin{equation}
\mathcal{L}(\theta)=\mathbb{E}_{\langle s_t,a_t,r_t,s_{t+1}\rangle\sim
D}\left[{\left(r_t+y^{\scriptscriptstyle TD}_{\scriptscriptstyle
DQN}\right)}^2\right]
\end{equation}
where
\[
y^{\scriptscriptstyle TD}_{\scriptscriptstyle DQN}=\gamma
\max_{a\in\mathcal{A}}Q(s_{t+1,a};\theta^{-})-Q(s_t, a_t;\theta),
\]
$\mathcal{D}$ is the Experience Replay buffer
\citep{lin1992self} and $\theta^-$ are the parameters of a frozen
target network used to stabilize value estimates. The use of this
different set of parameters is related to the bootstrapping error,
further explained in Section~\ref{sec:BE}.

% TS: Offline RL and difference with online RL
% NOTE (maybe) place this after bootstrapping error or any other
% section that introduces off-policy RL, so that the link is very
% clear between off-policy and offline
Finally, in \textit{offline} RL - also known as \textit{batch} RL -
the proposed MDP remain valid, but the agent loses the ability to transition
from state $s_t$ to state $s_{t+1}$ by \textit{actively} choosing and
performing action $a_t$. Instead, an offline RL agent is given a
logged datasets $\mathcal{B}$ of experience tuples
$\langle s_t,a_t,s_{t+1},r_{t+1}\rangle$ generated by a \textit{behavior
policy} $\pi_{\beta}$, and its task is to learn a (possibly better)
policy than the latter from these trajectories.
Since learning occurs under a state-action distribution induced by a
policy different from the current policy $\pi$, offline RL is also
known as \textit{fully off-policy} RL, and an offline agent needs to
maximize data exploitation because it lacks the possibility to
explore.

\subsubsection{The Off-Policy Bootstrapping Error}\label{sec:BE}
Off-policy RL is systematically afflicted by a source of error denoted
as \textit{extrapolation error} \citep{pmlr-v97-fujimoto19a}. Due to a
mismatch between the state-action distribution induced by the current
policy and the one contained in the batch, the $Q$ function is unable
to correctly estimate the value of unseen state-action pairs. As a
result, such inputs receive artificially high estimates which
skew the $Q$ function, and possibly cause it to
diverge. \citet{pmlr-v97-fujimoto19a} remark that, when combined with
RL algorithms which employ a maximization operator to compute
$Q^*\left(s,a\right)$ like Q-learning, the extrapolation error
induces a consistent positive \textit{overestimation bias}
\citep{thrun1993issues} in the $Q$ function.

The bootstrapping error \citep{kumar2019stabilizing} is a form of
extrapolation error which appears in algorithms that bootstrap to
compute their targets. These algorithms create the true target of a
regression problem using their own current estimate of such target,
which is a biased estimator. Referring to update rules~\ref{eq:ql_td}
and~\ref{eq:dqn_td}, it is clear that Q-learning-based algorithms are
prone to the bootstrapping error since their respective TD-targets
both come from a present estimate of $Q$. This is why, in practice,
deep RL algorithms keep a copy of frozen parameters $\theta^-$ which are
updated at intervals and used to compute the target estimates,
hence simulating a second static estimator that is not regressing
towards itself.
Given that the TD targets are
arbitrarily wrong during training, maximizing the $Q$-values with
respect to actions at the next state as in Equation~\ref{eq:dqn_td}
might
evaluate the $Q$ function on actions that do not correspond to the
training data distribution. Since such out-of-distribution (OOD)
actions \citep{kumar2019stabilizing} are not contained in the training
batch, their true value is unknown; a naive maximization will then
pick these overestimated $Q$-values, therefore compounding and
propagating the bootstrapping error during training through Bellman
backups. In the most extreme case where the $Q$ function is
initialized with high positive values only at OOD actions, a
Q-learning based agent will thus learn to perform these very actions
and disregard information gathered from the behavior policy
$\pi_\beta$.

The bootstrapping error is especially detrimental in offline RL, where
no additional data collection is possible. In the online case, the
wrong estimation of $Q\left(s,a\right)$ for some $\left(s,a\right)$
pair can be adjusted by actually performing $a$ and assessing its
result. However, the dataset $\mathcal{B}$ used by an offline RL agent
is fixed and no further exploration is possible. As a consequence,
dealing with bootstrapping error is crucial for the success of offline
RL algorithms.


\subsubsection{Bootstrapping Error
Correction}\label{sec:intro_related_work}
Techniques to correct bootstrapping errors or, more generally, to
minimize distributional shift in offline reinforcement learning
involve either \textit{policy constraint} or
\textit{uncertainty-based} methods \citep{levine2020offline}.

The $Q$ functions is evaluated on the same states that it is
trained on. Therefore, only the action inputs across states can be out
of distribution in the training process. \textit{Policy constraint}
methods address this issue by bounding the distribution over actions
used for the computation of the TD-targets,
$\pi\left(a^\prime\middle|s^\prime\right)$, to stay in the proximity
of the one induced by the behavior policy,
$\pi_\beta\left(a^\prime\middle|s^\prime\right)$. In this way, the $Q$
function regression is driven by target values for which enough
reliable information is found in $\pi_\beta$.
The difference
between these techniques resides in the metrics they employ to define
distributional proximity.
For example, Batch-Constrained deep Q-Learning (BCQ)
\citep{pmlr-v97-fujimoto19a} trains a generative model -- a variational
auto-encoder (VAE) \citep{kingma2013auto} -- to produce actions which
are likely given the data in $\mathcal{B}$, then it perturbs them to
increase diversity in a constrained manner, and finally selects the
maximum $Q$-value over these artificial actions. By substituting the
maximum over all possible actions at the next state, involved in the
computation of the TD-target, with the maximum over actions likely
under $\pi_\beta$ BCQ ensures that the learned policy $\pi$ is
centered around $\pi_\beta$, hence that the $Q$ function is not
queried on OOD actions with resulting bootstrapping
error. Bootstrapping Error Accumulation Reduction Q-Learning (BEAR-QL)
\citep{kumar2019stabilizing} also follows the intuition of placing
constraints on the learned action distribution, however it achieves so
with fewer restrictions. This makes it more viable for $\pi$ to
improve on $\pi_\beta$, a capability hindered by the aggressive
constraint of BCQ.\ Instead of requiring the learned policy to be
close in distribution to $\pi_\beta$, BEAR demands a \textit{support
constraint} \citep{kumar}. This loose condition means that the
learned policy must place non-zero probability on all those actions
that have non-negligible probability according to the behavior policy.
This precaution allows BEAR to improve over suboptimal, even random
off-policy data, whereas BCQ would learn a policy close to uniform
\citep{kumar2019stabilizing}.
